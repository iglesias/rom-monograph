\documentclass[a4paper]{article}

\usepackage{graphicx}           % For including graphics.
\usepackage{hyperref}           % To use some cool links within the report.

\addtolength{\topmargin}{-20mm}% Margin adjustments.
\addtolength{\textheight}{20mm}% Margin adjustments.

\begin{document}

\newcommand{\batman}{B.A.T.M.A.N.}

\title{Mobile Computers Networks \\
Monograph on \batman}
\author{Fernando J. Iglesias Garc\'{i}a \\ fjig@kth.se}

\maketitle

% TODO Index

% TODO Get rid of the numbers here
\section{Motivation}

First of all, some of the reasons why the Better Approach To Mobile
Ad-hoc Networking (\batman) routing protocol is interesting to study are
presented. The focus is mainly on presenting the protocol as an alternative
to the Optimized Link State Routing (OLSR) protcol rather than to motivate
the usefulness of mobile Ad-hoc networks. \\

\vspace{2mm}
OLSR, as any other link-state routing protocol, relies on the fact that the
nodes share roughly the same graph of the network. The more this information
differs among nodes, the more likely it is that \emph{bad things}
happen while the nodes build their routing tables. This observation becomes an
important issue in mobile or lossy networks where the complexity of maintaining
synchronized the information of the network between nodes increases
exponentially with the number of nodes. Consequently, it may be reasonable to
address the routing problem from a different perspective to link-state
algorithms. Moreover, the actual OLSR daemon comes with a bunch of extensions
that, at least in the opinion of \batman's development group, try to separate
precisely the protocol from a traditional link-state protocol's 
design~\cite{whyBatman}. \\

\vspace{2mm}
In addition to this and related to the last part discussed in the paragraph
above, there is no fully working implementation of OLSR as specified in RFC 3626
for real scenarios~\cite{whyBatman}. Nevertheless, the implementation of \batman
is not completely consistent with its specification
either~\cite{mailBatmanComments} but it must be taken into account that in this
case the specification is basically a draft~\cite{batmanRFC} and is expected 
to evelop meanwhile the protocol is developed.

\section{Philosophy of the Protocol}

To start with the description of the routing protocol, it is required first to
classify it as distance-vector routing protocol or, on the contrary, as
link-state routing protocol~\cite{applet-DVvsLS}. \batman is better found within
the first group, distance-vector routing protocol, since the knowledge of the
network is not found in each and every one of the nodes completely but
distributed. Every node is just concerned about which is the best neighbour
towards each of the other non-neighbour nodes of the
network~\cite{pdf-batman-status}. \\

\vspace{2mm}
A somewhat distinguishing feature of the flooding mechanism described
below~\ref{sec:protocol-description} is that it is timeless.
In other words, there are no timeouts for the topology information of the
network. Apart from that, the algorithm is proactive, the nodes look for routes
actively and they do not wait until they are required, versus reactive. Another
example of proactive protocol is OLSR, Ad-hoc On-demand Distance Vector (AODV)
is an example of a reactive protocol for mobile Ad-hoc networks.

\section{Description of the Protocol}
\label{sec:protocol-description}

The \batman protocol algorithm is briefly described here. In common with the
operation of other routing protocols, \batman nodes start their operation with
the transmission of broadcast messages. In \batman's particular frame, these
are known as originator messages, namely OGMs. The neighbours of the node
continue transmitting in broadcast the OGMs attending to some rules that are
described below. Rapidly, the network gets full of OGMs. It stems from this
flooding the requirement for the format of OGM messages to be small enough so
the protocol can be as small and fast as possible~\cite{batmanProtocol}.  \\

\vspace{2mm}
One node can re-broadcast OGMs either zero or one time, the messages are
uniquely identified using the \emph{Originator Address} and \emph{Sequence
Number}~\footnote{ Each node is reponsible for generating OGMs with different
sequence numbers. According to the RFC draft, this is achieved incrementing by
one consecutively the sequence number} fields~\cite{batmanRFC}. The other 
requirement for each OGM to be re-broadcast is that the message must have been 
received from the neighbour which is identified as the best next step towards 
the originator of the OGM~\cite{batmanProtocol}. \\

\vspace{2mm}
\batman nodes make use of a bit of intelligence in order to choose which is the
best neighbour in case that there exists more than one neighbour. Looking at the
statistics of the received OGMs, in particular to the time it took to receive
the messages from each of the neighbours and how reliable they were received,
the nodes make the decision of saving one or another neighbour in their routing
table as the best next hop~\cite{batmanProtocol}. \\

\vspace{2mm}
The first version of the protocol made the big assumption that all the links
where bidirectional, i.e. if then node A receives a OGM from B, then A
considered directly that B was accesible. In some kind of networks, radio over
all, it is not strange to find that communication is only possible in one way.
The second version of the protocol stopped with this consideration introducing
the so-called \emph{Bidirectional Link Check}~\cite{batmanWikipedia}. This 
check is basically  another step in the flooding mechanism and it is implemented 
using information of the Sequence numbers of past and self-originated OGMs and 
another flag of the OGM which has not been introduced in this 
monograph~\cite{batmanRFC}.  \\

\vspace{2mm}
In addition to Originator Messages (OGMs) there is another type of messages
called Host Network Announcement (HNA). The HNAs are optional and provide
extensions in the semantics of the inmediately before preceding OGMs. They are
used to announce a gateway to a network or host. For example, if there is a
\batman mobile Ad-hoc network in which one node has external access to internet
in another interface, this node could work as a gateway to the outer internet 
for all the other nodes in the mobile network~\cite{batmanRFC}.  \\

\section{The Visualization Server}

The vis server~\cite{visServer} is a very useful and interesting graphical tool 
to use in order to gain insight of the operation of the protocol and to debug in
development. A method for visualizing the state of the network is especially 
appealing for routing protocols such as \batman where each node gathers 
information about the best next hop. In other words, when no node has a full
picture of how the network looks like~\footnote{ Note that although it is useful
for us to picture and think of the graph of the network as a whole and not as 
something distributed, the latter is the most interesting implementation owing
to the less amount of data in the communication and stability, among other
reasons}. The vis servers can be connected to other tools for mesh visualization to 
generate fancy three dimensional graphics, Figure~\ref{fig:vis-server}.

\begin{figure}[!ht]
  \centering
  \includegraphics{pictures/08-vis_zoomed.eps}
  \caption{Graphic representation of a network generated using the vis server
    together with s3d tools.}
    % TODO reference the image
  \label{fig:vis-server}
\end{figure}

% TODO Format output by the visualization server

\section{\batman  advanced}

The original implementation of \batman, the so-called batman daemon, together
with most of the wireless routing protocols is built on layer 3 (transport
layer). This means that the protocol makes use of the User Datagram Protocol 
(UDP) to exchange datagrams and the routing algorithm is deployed on it. 
\batman advanced, commonly known as batman-adv, is an alternative implementation 
of the protocol built on layer 2 (link layer)~\cite{batman-adv}.  \\

\vspace{2mm}
The module batman-adv is included in the Linux kernel from the version
2.6.38~\cite{batman-adv-kernel}. The module had to be developed as a kernel
driver because it is not possible to neither to route nor to forward ethernet
packages. \\

\vspace{2mm}
One interesting point of batman-adv is that it turns out to be very light
loaded, it fully operates on the link layer using raw ethernet packages to
transport the routing information. On the contrary to UDP, batman-adv can be
thought of a transport protocol based on virtual networks. This approximation
offers interesting properties for mobile networks such as unawareness of the
topology of the network and robustness against network network changes. \\

\vspace{2mm}
Currently, only the branch of the project that works on layer 2 remains active
and the initial one, batman daemon, is regarded as unmaintained.

\begin{thebibliography}{99}

  % TODO Author in the RFC references, order references

  \bibitem{pdf-batman-status} Neumann, C., "Elektra" Aichele, C., Lindner, M.
  June 28, 2007. \emph{\batman Status Report}. 

  \bibitem{applet-DVvsLS} \emph{A Java applet implementing the Distance Vector
  and the Link-State algorithms for pedagogical purposes.} Retrieved on December
  5, 2011.

  \bibitem{whyBatman} Why starting \batman? Retrieved on December 4, 2011
  from http://www.open-mesh.org/wiki/batmand/Why-starting-batman. 

  \bibitem{olsrRFC} RFC 3626 \emph{Optimized Link State Routing Protocol (OLSR)}.
  http://www.ietf.org/rfc/rfc3626.txt.

  \bibitem{mailBatmanComments} J. Chroboczek. \emph{A few comments on the BATMAN
  routing protocol}.  Retrieved on December 4, 2011 from
  http://lists.alioth.debian.org/pipermail/babel-users/2008-August/000151.html.

  \bibitem{batmanRFC} RFC Draft \emph{Better Approach To Mobile Ad-hoc
  Networking (\batman)}.
  http://tools.ietf.org/html/draft-wunderlich-openmesh-manet-routing-00

  \bibitem{batmanProtocol} \emph{\batman protocol concept}. Retrieved on
  December 4, 2011 from http://www.open-mesh.org/wiki/open-mesh/BATMANConcept.

  \bibitem{visServer} \emph{Visualize the mesh}. Retrieved on December 5, 2011
  from http://www.open-mesh.org/wiki/batmand/VisualizeMesh.

  \bibitem{batman-adv} \emph{\batman advanced}. Retrieved on December 5, 2011
  from http://www.open-mesh.org/wiki/batman-adv/Wiki.

  \bibitem{batman-adv-kernel} \emph{Linux 2.6.38}. Retrieved on December 5,
  2011 from
  http://kernelnewbies.org/Linux\_2\_6\_38\#head-17577655766f585c3c47df886fe91dba276f4c3f.

  \bibitem{batmanWikipedia} \emph{Wipedia on \batman}. Retrieved on December 5,
  2011 from http://en.wikipedia.org/wiki/B.A.T.M.A.N.

\end{thebibliography}
\end{document}
